\documentclass[letter,10pt]{article}
\usepackage[utf8]{inputenc}
\usepackage{todonotes}
\usepackage{graphicx}
\usepackage{hyperref}
\hypersetup{
    colorlinks,
    linkcolor={red!50!black},
    citecolor={blue!50!black},
    urlcolor={blue!80!black}
} %TODO: Check colors
\usepackage{nameref}

%http://tex.stackexchange.com/questions/60209/how-to-add-an-extra-level-
%of-sections-with-headings-below-subsubsection
\makeatletter
\renewcommand\paragraph{\@startsection{paragraph}{4}{\z@}%
            {-2.5ex\@plus -1ex \@minus -.25ex}%
            {1.25ex \@plus .25ex}%
            {\normalfont\normalsize\bfseries}}
\makeatother
\setcounter{secnumdepth}{4} % how many sectioning levels to assign 
%numbers to
\setcounter{tocdepth}{4}    % how many sectioning levels to show in ToC

%opening
\title{Automating Code Magnet Generation}
\author{Julia Dana}
\date{}

\begin{document}

\maketitle

%\begin{abstract}
%\end{abstract}

\pagebreak
\tableofcontents

\pagebreak

\section{Introduction}

\subsection{The Overview: Easier Creation of Code Magnet Microlabs}

The purpose of this project is to assist in the creation of code magnet 
microlab assignments for WAGS by creating magnets from a completed 
solution file. This is accomplished in a manner that supports multiple 
programming languages, and allows additional languages to be added with 
minimal configuration. Additional tools that support this idea of 
easier creation of assignments are also included, such as an automated
interaction with the WAGS website to create assignments. Also, this 
project defines new formats for the specification of magnets, including 
object, JSON and YAML. \todo{Improve this sentence}

\subsection{The Context: What is WAGS?}

WAGS (Web Automated Grading System) is an ongoing project of 
Appalachian State University. It is an online tool for microlabs. 
Microlabs are short, 5-10 minutes hands-on activities that are 
intended to be done as a part of a regular (i.e. not lab) class session 
to reinforce the concepts that are currently being covered. There 
are multiple types of microlabs provided by WAGS, but the one that 
this project is interested in is code magnet microlabs. These are 
microlabs where the student is given pieces of code (code magnets) and 
has to choose the correct ones and order them correctly to “write” a 
solution to the microlab.

\missingfigure{Add figure of in progress magnet microlab}
%\begin{figure}
% \centering
% \includegraphics[width=3in]{./images/missing-image.png}
% \caption{An in-progress magnet microlab}
% \label{fig:magnets}
% %\todo{Add figure of in progress magnet microlab}
% %\todo{Make sure picture is in a good location in the text}
%\end{figure}

\subsection{The Problem: Brittle Input}

The problem is that creating these magnet microlab assignments on the 
WAGS website is a somewhat painful process. The current parser creates 
a magnet per line, and this can result in having to format the input 
file so that is no longer the same as a solution file, and is indeed no 
longer valid for its language.

\missingfigure{Parser example}

If your input cannot be handled by this brittle parser. WAGS does 
provide a manual input for magnets. However, using this manual input 
requires magnets to either be entered one at a time to the magnet 
creation wizard, or for the user to directly type the final magnet 
(including HTML escape sequences) to the parsed output section. 

\missingfigure{magnet assignment creation page}

\subsection{The Solution: Parsing by Grammar}

The solution to this is to use a more robust parser that is based on 
the grammar of a language, rather than simply splitting on line breaks. 
This project does this by using the ANTLR4 parser 
generator\cite{antlr-reference} and grammars for common 
languages\cite{antlr-grammars-project}.

\todo{Discuss grammars}

\todo{Describe parser generators (using ANTLR4 as example)}

\section{Development results and future extensions}

\subsection{What It Does: Internal Functions}

\subsubsection{Parses by Grammar}

\todo{Clarify this entire section}
The magnetizer \todo{magnetizer? find a better word} takes an input 
file, and parses it with the parser generated by ANTLR from the grammar 
specification. The result of this parsing is a parse tree, which can 
be represented graphically. An example follows.

For this simple Java file, Hello.java
\missingfigure{Listing: Hello.java}

The resulting parse tree is:

\missingfigure{Show a simple parse tree}
\label{fig:parseTree}

Each language supported by the magnetizer has an entry in a 
configuration file that specifies which nodes in the parse tree should 
trigger the creation of a magnet. Whenever a child node triggers its 
own magnet, the parent is given a drop zone for that section, and that 
section of text is not directly included in the parent magnet.


\todo{ANTLR visitor vs. listener}


\subsubsection{Alternative Magnets and Other Instructor Directives}

\subsubsection{Improved Represention of Magnets}

The old format for describing magnets is difficult to read and worse to 
try to manually type. Some of its quirks are that magnets are separated 
by an unusual separator (\verb~.:|:.~), areas that accept other magnets 
(aka drop zones) are indicated by a seemingly random set of HTML tags 
(\verb~<br><!-- panel --><br>~), and any special characters used in the 
code magnet must be escaped for HTML (so something like \verb~1 < 2~ 
would need to be entered as \verb~1 &lt; 2~). This format also is 
limited when trying to create more advanced types of magnets. The WAGS 
system addresses this problem by adding more form fields when adding 
``advanced Java magnets'', but this does not work well for adding 
additional languages.

\missingfigure{Creating a magnet assignment with advanced magnets}

This project creates an underlying data structure for magnets, as 
shown in figure \todo{link figure of class diagram}. This structure can 
then be serialized as desired. Currently, serialization to the old 
magnet format is supported for backward compatibility, as well as JSON 
and YAML for more robust usages.

\missingfigure{Class diagram of Magnets}


\subsection{What It Does: External Functions}

\subsubsection{Command Line Tool}

%\subsubsection{GUI Tool}

\subsubsection{Automated Interaction With WAGS Website}



\subsection{What It Could Do: Future Extensions}

\subsubsection{Automatic Generation of Alternative/Distractor Magnets}

The next step in easily creating code magnet assignments is to have the 
magnet creation tool not simply create magnets needed for the correct 
solution and any additional magnets specified by the instructor, but 
also to automatically create appropriate alternative ``distractor'' 
magnets for common student misconceptions and errors. This level of 
manipulation is available because we have the whole parse tree to work 
with during magnet creation, but would require significant work per 
language to define.

\subsubsection{GUI Tool}

A GUI tool that can open an input file, perform the actions of the CLI 
tool on it, and also has an editor to assist in the addition of any 
instructor directives desired would be a nice addition to this project.

\subsubsection{Supporting New Languages}

This project currently can create magnets for Python3 and Java, however 
it is set up to easily allow the addition of new languages. 
Basically, this is done by finding and adding an ANTLR4 grammar file 
for the desired language, making sure it conforms to a handful of 
guidelines, writing a short configuration file, and running a setup 
action on the project. Full details of this process are in 
section~\ref{sec:newLang} \nameref{sec:newLang}.


\subsubsection{Additional Serialization Formats}

Because magnets are now objects, additional serialization formats (such 
as XML) could easily be added. 



\section{User Guide}

\subsection{Quick Start - CLI Tool}

The primary interface to this project is the magnetizer command line 
tool. This tool is capable of outputing to WAGS-style magnets that can 
be copy-pasted into the parsed sections of the WAGS website or to a 
JSON or YAML representation of the magnets.

\todo{Make sure usage is the latest version.}
\begin{verbatim}
Usage: magnetizer [options] file
    -l, --language LANGUAGE          Specify a language (default: Java)
        --json                       Print the JSON output
        --yaml                       Print the YAML output
        --[no-]wags                  Print the output as WAGS magnets 
(default)
    -o, --output-file BASE_FILENAME  Output to a file
    -h, --help                       Show this message
\end{verbatim}


%\subsection{Quick Start - GUI Tool (tent.)}

% An overview of how a file is split into magnets.
\subsection{Explanation of Output}

\subsubsection{Java}
\todo{explain this in a way that doesn't require 
deep understanding of 
the ANTLR grammar.}
The current configuration for Java creates magnets for
\begin{itemize}
 \item Package declarations
 \item Import declarations
 \item Type declarations
 \item Class body declarations - this is the nonterminal that includes 
anything that can be directly in the class body
 \item Block statements - this is the nonterminal that includes 
anything that can be directly inside a block.
\end{itemize}



\subsubsection{Python 3}
\todo{explain this in a way that doesn't require deep understanding of 
the ANTLR grammar.}
The current configuration for Python creates magnets for
\begin{itemize}
 \item simple statements
 \item compound statements
\end{itemize}


\subsection{Automatic Upload to WAGS Website}

% Instructor directives p. 1
\subsection{Creating Alternative Magnets}

% All the other instructor directives
\subsection{Controlling the Output (tent.)}



\section{Developer Notes}

\subsection{The Environment: Languages and Libraries Used}

\todo{Section needs more details}

This project is written to run on JRuby. The magnetizer itself is 
written in Ruby, but it uses the Java classes provided by ANTLR. JRuby 
allows this to occur.

Automated interaction with the WAGS site is provided by using Capybara 
to interact with Selenium (drives Firefox) or Poltergeist/PhantomJS 
(headless).

Rake (Ruby make) is used to process new grammar files and other build 
tool functionality.

RSpec is used to handle testing

\subsection{The Testing}

\subsection{The Design}

\subsection{The Configuration: Specifying Magnet Sections}

\subsection{The Expansion: Adding a New Language}
\label{sec:newLang}

\subsubsection{New Grammar Specification}
A new G4 file can be added to the system. However, whitespace is
require to go on channel 1, or your magnets will not have any
whitespace, and you might get things like ``publicclassMyClass''.


\todo{Final export of bibliography from Mendeley and link it}
\todo{Make sure citations and bibliography are styled correctly}
\bibliography{bibliography}{}
\bibliographystyle{plain}

\end{document}
