\documentclass[letter,10pt]{article}
\usepackage[utf8]{inputenc}
\usepackage{todonotes}
\usepackage{graphicx}

%http://tex.stackexchange.com/questions/60209/how-to-add-an-extra-level-
%of-sections-with-headings-below-subsubsection
\makeatletter
\renewcommand\paragraph{\@startsection{paragraph}{4}{\z@}%
            {-2.5ex\@plus -1ex \@minus -.25ex}%
            {1.25ex \@plus .25ex}%
            {\normalfont\normalsize\bfseries}}
\makeatother
\setcounter{secnumdepth}{4} % how many sectioning levels to assign 
%numbers to
\setcounter{tocdepth}{4}    % how many sectioning levels to show in ToC

%opening
\title{Automating Code Magnet Generation}
\author{Julia Dana}
\date{}

\begin{document}

\maketitle

%\begin{abstract}
%\end{abstract}

\pagebreak
\tableofcontents

\pagebreak

\section{Introduction}

\subsection{The Overview: Easier Creation of Code Magnet Microlabs}

The purpose of this project is to assist in the creation of code magnet 
microlab assignments for WAGS by creating magnets from a completed 
solution file. This is accomplished in a manner that supports multiple 
programming languages, and allows additional languages to be added with 
minimal configuration. Additional tools that support this idea of 
easier creation of assignments are also included, such as an automated
interaction with the WAGS website to create assignments. Also, this 
project defines new formats for the specification of magnets, including 
object and JSON. \todo{Improve this sentence, implement XML}

\subsection{The Context: What is WAGS?}

WAGS (Web Automated Grading System) is an ongoing project of 
Appalachian State University. It is an online tool for microlabs. 
Microlabs are short, 5-10 minutes hands-on activities that are 
intended to be done as a part of a regular (i.e. not lab) class session 
to reinforce the concepts that are currently being covered. There 
are multiple types of microlabs provided by WAGS, but the one that 
this project is interested in is code magnet microlabs. These are 
microlabs where the student is given pieces of code (code magnets) and 
has to choose the correct ones and order them correctly to “write” a 
solution to the microlab.

~

~

\missingfigure{Add figure of in progress magnet microlab}
%\begin{figure}
% \centering
% \includegraphics[width=3in]{./images/missing-image.png}
% \caption{An in-progress magnet microlab}
% \label{fig:magnets}
% %\todo{Add figure of in progress magnet microlab}
% %\todo{Make sure picture is in a good location in the text}
%\end{figure}

\subsection{The Problem: Brittle Input}

The problem is that creating these magnet microlab assignments on the 
WAGS website is a somewhat painful process. The current parser creates 
a magnet per line, and this can result in having to format the input 
file so that is no longer the same as a solution file, and is indeed no 
longer valid for its language.

\missingfigure{Parser example}

If your input cannot be handled by this brittle parser. WAGS does 
provide a manual input for magnets. However, using this manual input 
requires magnets to either be entered one at a time to the magnet 
creation wizard, or for the user to directly type the final magnet 
(including HTML escape sequences) to the parsed output section. 

\missingfigure{magnet assignment creation page}

\subsection{The Solution: Parsing by Grammar}

The solution to this is to use a more robust parser that is based on 
the grammar of a language, rather than simply splitting on newlines. 
This project will be using the ANTLR4 parser 
generator\cite{antlr-reference} and grammars for common 
languages\cite{antlr-grammars-project}.

\todo{Discuss grammars}

\todo{Describe ANTLR4}

\section{Development results and future extensions}

\subsection{What It Does: Internal Functions}

\subsubsection{Parses by Grammar}
\paragraph{Java}
\paragraph{Python}

\subsubsection{Improved Represention of Magnets}

\subsubsection{Alternative Magnets}


\subsection{What It Does: External Functions}

\subsubsection{Interaction With WAGS Website}

\subsubsection{GUI Tool}


\subsection{What It Could Do: Future Extensions}

\subsubsection{Supporting New Languages}

\subsubsection{Automatic Generation of Alternative Magnets}



\section{User Guide}

\subsection{Quick Start - CLI Tool}

\subsection{Quick Start - GUI Tool (tent.)}

\subsection{Explanation of Output}

\subsection{Automatic Upload to WAGS Website}

\subsection{Creating Alternative Magnets}

\subsection{Controlling the Output (tent.)}



\section{Developer Notes}

\subsection{The Environment: Languages and Libraries Used}

\subsection{The Design}

\subsection{The Configuration: Specifying Magnet Sections}

\subsection{The Expansion: Adding a New Language}

\subsubsection{New Grammar Specification}
A new G4 file can be added to the system. However, whitespace is
require to go on channel 1, or your magnets will not have any
whitespace, and you might get things like ``publicclassMyClass''.


\todo{Final export of bibliography from Mendeley and link it}
\todo{Make sure citations and bibliography are styled correctly}
\bibliography{bibliography}{}
\bibliographystyle{plain}

\end{document}
